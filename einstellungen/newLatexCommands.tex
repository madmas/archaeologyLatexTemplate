%
% --| Layout/Structure |--------------------------------------------------
%
\renewcommand\newline{\smallskip \\} %Neue Zeile
\newcommand\newpar{\smallskip \par} %Neue Zeile
%
%
\newcommand{\marginlabel}[1]{\mbox{}\marginline{\hspace{0pt}\RaggedRight\footnotesize #1}}
%
% -------------------------------------------------------------------------

% --| Math |-------------------------------------------------------
% -- new commands --
\providecommand{\abs}[1]{\lvert#1\rvert}
\providecommand{\Abs}[1]{\left\lvert#1\right\rvert}
\providecommand{\norm}[1]{\left\Vert#1\right\Vert}
\providecommand{\Trace}[1]{\ensuremath{\Tr\{\,#1\,\}}} % Trace /Spur
%

% -- differentials --
\renewcommand{\d}{\partial\mspace{2mu}} % partial diff
\newcommand{\td}{\,\mathrm{d}}	% total diff
\newcommand{\ddt}[1]{\frac{\td #1}{\td t}}

% -- Abbrevitations --
%\newcommand{\complex}{\mathbb{C}} % Complex
%\newcommand{\real}{\mathbb{R}}    % Real
%\newcommand{\N}{\mathbb{N}}
%\newcommand{\Z}{\mathbb{Z}}

%
%\newcommand{\Ham}{\mathcal{H}}    % Hamilton
%\newcommand{\unity}{\mathds{1}}   % Real
%

% -- New Operators --
%\DeclareMathOperator{\rot}{rot}
%\DeclareMathOperator{\grad}{grad}
%\renewcommand{\div}{\text{div}\,}
%\DeclareMathOperator{\Tr}{Tr}
%\DeclareMathOperator{\e}{e} 			% exponatial Function

% -- new symbols --
\newcommand{\laplace}{\Delta}
\newcommand{\dalembert}{\Box}


% --| Physics |--------------------------------
% (adopted from phystex by Florian Jung)
\newcommand\evec[1]{{\Hat{\vec{#1}}}}
\newcommand\op[1]{{\hat{\mathrm{#1}}}}  % Operator
\newcommand\vop[1]{{\Hat{\mathbf{#1}}}} % Vector Operator
%
\newcommand\bra[1]{\ensuremath{\left\langle{#1}%
	\mspace{1mu}\right\rvert\mspace{1mu}}} % Bra
%
\newcommand\ket[1]{\ensuremath{\mspace{1mu}\left\lvert\mspace{1mu}%
	{#1}\right\rangle}} % Ket
%
\newcommand\bracket[2]{\ensuremath{%						% Bra-Ket < | >
    \left\langle{#1}\mkern1.2mu\vert\mkern1.2mu{#2}\right\rangle}}
%
\newcommand\ketbra[2]{\ensuremath{%							% Ket-Bra | >< |
	\left\lvert\mspace{1mu}{#1}\right\rangle % Ket
	\mspace{-2mu}%
	\left\langle{#2}\mspace{1mu}\right\rvert % Bra
}}
%
\newcommand\expect[1]{\ensuremath{\left\langle{#1}\right\rangle}} %
\newcommand\smallexpect[1]{\ensuremath{\langle{#1}\rangle}} % Expectationvalue
%
\newcommand{\mean}[1]{\ensuremath{\overline{#1}}} % mean value
%
\newcommand{\state}[1]{\ensuremath{\ket{#1}}}
%
\newcommand\commutator[2]{\ensuremath{\mathinner{%
    \mathopen[\,#1,#2\,\mathclose]}}}
\newcommand{\Commutator}[2]{\ensuremath{\left[\,#1,#2\,\right]}}
\newcommand{\bigcommutator}[2]{\ensuremath{\bigl[\,#1,#2\,\bigr]}}
\newcommand{\Bigcommutator}[2]{\ensuremath{\Bigl[\,#1,#2\,\Bigr]}}
%
\newcommand\poisson[2]{\mathinner{%
    \mathopen\{#1,#2\mathclose\}}}
%


% --| Quantum Optics |-------------------------------------------------------

% -- Abbrevitations --
% \newcommand{\hc}{\,h.\,c.\,}
%\newcommand{\cc}{\quad c.\,c.\,}


% ------------------------------------------------------------------------------
% Umbenennungen von englischen �berschriften

%\renewcommand{\glossaryname}{Glossar} % benennt das Glossar vom englischen Titel "`Glossary"' nach "`Glossar"' um
%\renewcommand{\acronymname}{Abk"urzungsverzeichnis}

% ------------------------------------------------------------------------------
% Schriftanpassungen

% Schrift von Description-Umgebung auf normale Schrift aber fett setzen
\setkomafont{descriptionlabel}{\normalfont\bfseries}

%Schriften�nderungen: KOMA benutzt f�r die �berschriften eine serifenlose Schrift. Ich finde die Mischung- vor allem auf der Ebene subsubsection, wo die Schriftgr��e der �berschrift die gleiche wie die des Flie�textes ist, jedoch nicht sch�n. Nur diese �berschrift �ndern ist auch nicht sch�n, deswegen werden alle �berschriften in ihrer voreingestellten Gr��e, fett und *mit* Serifen gesetzt - in der normalen Schriftart.

\setkomafont{chapter}{\normalfont\bfseries\huge}
\setkomafont{section}{\normalfont\bfseries\Large}
\setkomafont{subsection}{\normalfont\bfseries\large}
\setkomafont{subsubsection}{\normalfont\bfseries}

%\setkomafont{pagehead}{\small\sffamily\slshape}        % Kopfzeile
%\setkomafont{pagenumber}{\bfseries\sffamily}             % Seitenzahl
%\setkomafont{sectioning}{\sffamily} %\rmfamily\bfseries  % Titelzeilen
%\setkomafont{caption}{\small}                          % Schrift f�r Caption
\setkomafont{captionlabel}{\normalfont\bfseries\small}   % Schrift f�r 'Abbildung' usw.

% Glossartitel
%\renewcommand{\entryname}{Kürzel}
%\renewcommand{\descriptionname}{Beschreibung}


% --- Anpassungen fürs Literaturverzeichnis

   % % Change Layout of Backref
%   \renewcommand*{\backref}[1]{%
   	% default interface
   	% #1: backref list
   	%
   	% We want to use the alternative interface,
   	% therefore the definition is empty here.
%   }%
%   \renewcommand*{\backrefalt}[4]{%
   	% alternative interface
   	% #1: number of distinct back references
   	% #2: backref list with distinct entries
   	% #3: number of back references including duplicates
   	% #4: backref list including duplicates
   	
   	
%   	\ifnum#1>0                   % <---
%   	\mbox{
%   	(Zitiert auf %
%   	\ifnum#1=1 %
%		   Seite~%
%	   \else
%   		Seite~%
%   	\fi
%   	#2)
   	%}
%   	\fi                              % <----
   	%
%   }


%
% WORKAROUND, damit lstlistoflistings funktioniert:
% Quelle: http://www.komascript.de/node/477
%
\makeatletter% --> De-TeX-FAQ
\renewcommand*{\lstlistoflistings}{%
\begingroup
\if@twocolumn
\@restonecoltrue\onecolumn
\else
\@restonecolfalse
\fi
\lol@heading
\setlength{\parskip}{\z@}%
\setlength{\parindent}{\z@}%
\setlength{\parfillskip}{\z@ \@plus 1fil}%
\@starttoc{lol}%
\if@restonecol\twocolumn\fi
\endgroup
}
\makeatother% --> \makeatletter
