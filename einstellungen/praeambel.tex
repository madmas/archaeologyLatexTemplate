% ------------------------------------------------------------------------
% Latex - Einstellungen **************************************************
% ------------------------------------------------------------------------
% ========================================================================


%
% - �ndern von Schriftschnitten - (Muss ganz am Anfang stehen !) -----
\usepackage{fix-cm}
% Siehe auch Dokumentation: http://home.vr-web.de/was/x/fix-cm.pdf
% 
%---------------------------------------------------------------------
\usepackage{calc} % Zum Rechnen innerhalb von Latex
% ----------------------------------------------------------------------------------------------

%EURO
\usepackage{eurosym}

% --- Schriften ---

\usepackage{mathpazo}
\usepackage{helvet}
\usepackage{courier}

% ----------------------------------------------------------------------------------------------

% --- Bild- und Tabellenunterschriften --- 
\usepackage[margin=10pt, font=small, labelfont=bf, format=plain]{caption}

% ----------------------------------------------------------------------------------------------

% --- Sprache --- 
\usepackage[german]{babel}
\usepackage[T1]{fontenc}
%\usepackage[latin1]{inputenc}
\usepackage[utf8]{inputenc}

%----------------------------------------------------------------------------------------------


%% Seitenlayout ================================================================================

% --- Zeilenabstand --- 
\usepackage{setspace}
\onehalfspace        % 1,5-facher Abstand  


% --- Kopfzeilen ---
% Linien auf alle Seiten oben und unten um Kopf- und Fu�zeilen abzutrennen
\usepackage[headsepline, plainheadsepline, footsepline, plainfootsepline]{scrpage2}

% Seite mit Headern
\pagestyle{scrheadings}
%
% l�scht voreingestellte Stile
\clearscrheadings
\clearscrplain
%
% Was steht wo...

\ohead[\headmark]{\headmark}
\rofoot[\pagemark]{\pagemark}

\automark{chapter}

\setlength{\headheight}{1.5\baselineskip}


% --- Linien ---
\setheadsepline{.4pt}
\setfootsepline{.4pt} 

% ----------------------------------------------------------------------------------------------


% --- Inhaltsverzeichnis --- 
\setcounter{secnumdepth}{3}  % Abbildungsnummerierung mit größerer Tiefe
\setcounter{tocdepth}{2}	% Inhaltsverzeichnis mit größerer Tiefe

% ----------------------------------------------------------------------------------------------


%% Bilder und Graphiken ======================================================================

% --- Bilder --- 
\usepackage{graphicx}

%
% --- ps4pdf --- 
%
% This package allows the use of Postscript commands in PDF Mode (pdflatex)
% Therefore the Command %\PSforPDF{...} is requibefore each PS-using Command
% i.e. all includefigure, color-commands, and the corresponding usepackages
%
%\usepackage{ps4pdf}

% --- Bilder einbilden --- 
%
\usepackage{subfigure}        % Bilder nebeneinander
% \usepackage{subfig}
\usepackage[rflt]{floatflt}   % Stellt Befehlt 'floatingfigure' zur Verf�gung
                              % [rflt] - Standard float auf der rechten Seite
\usepackage{wrapfig}

%~~~~~~~~~~~~~~~~~~~! PSforPDF ! ~~~~~~~~~~~~~~~~~~~~~~~~~~~~~~~~~~~~~~~~~~~
%\PSforPDF{
% *** Graphen ******************************
%%\usepackage{pst-plot}
%\usepackage{psfrag}
%-------------------------------------------
%} % End PSforPDF
%~~~~~~~~~~~~~~~~~~~! PSforPDF ! ~~~~~~~~~~~~~~~~~~~~~~~~~~~~~~~~~~~~~~~~~~~
%----------------------------------------------------------------------------------------------


%% Mathe =======================================================================================

% --- Mathematik ---
%
% \usepackage[sumlimits,intlimits,namelimits]{amsmath}


% --- Symbole ---
\usepackage{amssymb}
% ----------------------------------------------------------------------------------------------

%% sonstige Pakete =============================================================================
%
% --- Glossar ---
%\usepackage[style=long,border=none,header=plain,cols=2, number=none, hypertoc=true, acronym=false, global=true]{glossary}   %
%\setacronymnamefmt{\gloshort}  % setzt den ersten Wert im Abk�rzungsverzeichnis auf die Abk�rzung selbst
%\setacronymdescfmt{%\glolong: %\glodesc}
%\makeacronym
%\makeglossary


% Entweder alle Glossareintr�ge in eine extra Datei oder in die normalen Text-Dateien
%\input{glossar/glossar}

%\usepackage{makeidx}	% Index
% \usepackage{minitoc}	% Inhaltsverzeichnis vor jedem Kapitel

% --- Listen --- 
\usepackage{enumerate}  % Optionen [a)], [i)] usw.
\usepackage{mdwlist}	% Geringerer Abstand zwischen den Punkten
%
% --- Tabellen --- 
\usepackage{array}
\usepackage{tabularx}   % Erweiterte Tabellen Optionen
\usepackage{booktabs}

\usepackage{lscape}

% --- Dokument --- 

% Besseres Darstellen der URLs: Umbruch m�glich, verlinkt mit hyperref-Package
\usepackage[hyphens]{url}
\urlstyle{same} % normale Schriftart. Wenn das weggelassen wird, wird eine tt-Schriftart benutzt
\usepackage{soul}		% Unterstreichen, Sperren

\usepackage{color} % Farbiger Text 
\definecolor{DarkBlue}{rgb}{0.0,0.0,0.5} 
\definecolor{DarkRed}{rgb}{0.5,0.0,0.0} 

% --- PDF --- 
\usepackage[pdfborder=false]{hyperref}  % Stellt Verlinkungen im Dokument her (Inhaltsverzeichnis, Quellenangaben, etc.), ohne einen Rand um die Links zu ziehen (und ohne sie farbig zu machen)
%
\hypersetup{    pdfauthor={Autorname},
                pdftitle={Titel des Dokuments},
                pdfsubject={},
                pdfproducer={},
	 	pdfkeywords={},
		pdfpagelayout=TwoColumnLeft,
                 linkcolor=black,        %% for links on same page
                 citecolor=black, 
                pagecolor=black,          %% for links to other pages
                urlcolor=black,          %% links to URLs
%                 breaklinks=true,        %% allow links to break across lines!
                 colorlinks=true, %true, %false,       %% obviously inverted logic :-(
%                 citebordercolor=0 0.5 0,        %% color for \cite
%                 filebordercolor=0 0.5 0,
%                 linkbordercolor=0.5 0 0,
%                 menubordercolor=0.5 0 0,
%                 pagebordercolor=0.5 0.5 0,
%                 urlbordercolor=0 0.5 0.5,
%                 pdfhighlight=/I,
%                 pdfborder=0 0 0.5,      %% to show _really_ only a box
                                        %% around the links and not a
                                        %% filled recangle
%                 backref=true,
 %                pagebackref=true,
                 bookmarks=true,
                 bookmarksopen=false,
                 bookmarksnumbered=true
}

% --- System --- 
\usepackage{mparhack}
\usepackage{xspace}

%----------------------------------------------------------------------------------------------

% --- @ zu einem TeX Zeichen machen --- 
\makeatletter %%%%%%%%%%%%%%%%%%%%
% --------------------------------

% --- Caption f�r nicht-flie�enden Figure Umgebung --- 
\newcommand\figcaption{\def\@captype{figure}\caption}
\newcommand\tablecaption{\def\@captype{table}\caption}

% ----------------------------------------------------------------------------------------------
\usepackage{color}
\definecolor{Gray}{gray}{0.5}


\usepackage{listings}
\lstset{captionpos=b,frame=shadowbox,rulesepcolor=\color{Gray}}

%\normalcolor
\listfiles

\usepackage[citestyle=authoryear-comp,bibstyle=authortitle,sorting=nyt,dashed=false,%
    maxcitenames=1]{biblatex}

\newcounter{mymaxcitenames}
\AtBeginDocument{%
  \setcounter{mymaxcitenames}{\value{maxnames}}%
}

\renewbibmacro*{begentry}{%
  \printtext[brackets]{%
    \begingroup
    \defcounter{maxnames}{\value{mymaxcitenames}}%
    \printnames{labelname}%
    \setunit{\nameyeardelim}%
    \usebibmacro{cite:labelyear+extrayear}%
    \endgroup
    }%
  \quad% or \addspace
}

\DeclareNameAlias{sortname}{first-last}
%\DefineBibliographyStrings{german}{andothers={et\ al\adddot}} % aus u.a. zu et al. machen
%\DefineBibliographyStrings{ngerman}{and={\&}} % aus und zu & machen
\DefineBibliographyStrings{german}{and={--}}

% Literaturverzeichnis einfuegen, Verweis ins Inhaltsverzeichnis
\bibliography{literatur/literatur}
