\chapter{Beispielkapitel}

\section{Ein Abschnitt}

Lorem ipsum....wäre ein netter Platzhaltertext an dieser Stelle, steht uns aber gerade nicht zur Verfügung.

Referenzen auf andere Kapitel, wie zum Beispiel das zusammenfassende Fazit und Ausblick \ref{chap:fazit} erleichtern die Orientierung im Dokument. Ebenso lassen sich Bilder referenzieren, z.B. Abblidung \ref{fig:testbild}.

\subsection{Ein Unterabschnitt}

Anhand des Textes kann man sich ein Bild vom Textsatz und dem Schriftbild des Ergebnisdokuments machen. Möchte man Anführungsstriche verwenden, geht das \glqq am besten\grqq mit den entsprechenden Steuerzeichen \textbackslash glqq und \textbackslash grqq, da die \glqq normalen\grqq Anführungsstriche von \LaTeXe{} anders verarbeitet werden.\footnote{Wobei sich dieses Verhalten auch beeinflussen lässt. Dafür kann man an dieser Fussnote als Beispiel die Verwendung einer Fussnote sehen.}

Mit einem weiteren Absatz kann das Schriftbild noch besser veranschaulicht werden. Möchte man ein Zitat einfügen, kann man dies mit \textbackslash cite machen. Dieser Text könnte ein Zitat aus \cite{Demesnil.2010} sein.

Oft sollen Verweise jedoch auch in Fussnoten erscheinen, also kombiniert man cite mit footnote.\footnote{\cite{Thierry.1986}}. Ein Beispiel für eine Literaturangabe mit mehreren Autoren findet sich auch im Literaturverzeichnis.\footnote{Vgl. \cite{Berije.1981}}